\documentclass[11pt]{article}
\usepackage{times}
\thispagestyle{empty}
\begin{document}

\section*{Research Statement}
Clusters of workstations are rapidly emerging as cost-effective solutions for 
delivering high performance for many scientific applications. In these 
architectures, not only do the multiple CPUs provide for the processing 
parallelism, the multiple disks on each workstation can provide for the much 
needed storage-level parallelism for data access and transfer. Parallel file 
systems seek to harness the storage capacities and bandwidths of all the disks of a cluster 
and hide the I/O bottlenecks from such applications. 

My research interests have largely focused on caching and consistency
algorithms for parallel file-systems especially in the context of high-performance
scientific computing applications. With little or no information about the workloads to be supported,
traditional approaches to file-system designs have always adopted
a one-glove-fits-all approach in dealing with caching and consistency semantics.
Contrary to the conventional wisdom of separating mechanisms and policies for system
design, parallel file-systems impose caching policies and consistency semantics for all applications. 
Taking an inflexible stance on caching or consistency can demote throughput, latency and scalability
for such applications. My research addresses this important
problem of separating mechanisms and policies for caching and consistency policies
and using application-level access pattern information to exploit these mechanisms for efficient
performance. In many instances, such access pattern information can be automatically
gleaned by a compiler or inferred by the runtime system (middleware libraries) and in some
cases the onus is on application designers or system administrators to make use
of the mechanisms for better performance and scalability. 

While the parallelism 
offered by the numerous disks of the cluster can alleviate the I/O bandwidth 
problem, it does not really address the latency issue that is largely limited 
by seek, rotational and network transfer costs. Caching data blocks in main 
memory is a well-known way of reducing I/O latencies, provided we can achieve 
good hit rates. I/O caching is typically done in software, and the overheads 
of cache lookup and maintenance could become quite high. Further, we may need 
multiple levels of such caches to reduce capacity misses for such large 
data-sets. On such systems, the cost of going to the upper levels of the cache 
and not finding the data there (before going to disk) might be quite 
substantial. Consequently, it becomes important to intelligently determine what
blocks to place in caches and when to avoid or bypass the caches whose lookup 
costs are significantly higher on I/O requests. However, this largely depends 
on the application's data access patterns and hence we need compiler-based/static 
and runtime techniques that automatically try to identify frequently 
used blocks and place them on caches that are closer to the client nodes.
A detailed description and evaluation of our prototype ``discretionary caching'' system
built atop a popularly used parallel file-system (PVFS) on Linux clusters
was published in IEEE Cluster 2002 and IEEE/ACM CCGrid 2003.

High-bandwidth I/O continues to play a critical role
in the performance of numerous scientific applications that manipulate
large data sets.
With little or no information about the
workloads to be supported, a file system designer has to often make
a hard choice regarding the consistency policies which can affect performance
adversely in some situations.
Leaving the choice and granularity 
of consistency policies to the user at open/mount time provides an
attractive way of providing the best of all worlds. The level of sharing when 
viewed at a file granularity in parallel computing
environments is much higher than that observed in network file 
systems, making consistency more important.
Enforcement of such consistency can, however, conflict with performance
and scalability goals. To address some of these deficiencies, I designed and
implemented a file-store, CAPFS (Content
Addressable Parallel File System), that
allows the user to define consistency semantic policies at runtime.
A client-side plug-in architecture based on user-written plug-ins 
leaves the choice of consistency policies to the end-user. A detailed description
of our prototype system was published in USENIX FAST 2005.

Based on the above research that I have done so far, I see several interesting
directions that I would like to pursue. Parallel file-systems have long been the vehicle
for delivering the much needed performance due to their simplicity of interfaces,
setup and management. Although POSIX has been the standard of choice that most
disk-based file-systems adher to, it is becoming increasingly apparent that
it may not be appropriate from a performance standpoint in a parallel file-system environment. 
I have been involved with a working group that is advocating augmentations to the POSIX
interfaces and standards that would allow implementors more leeway in exploiting performance.
However, performance is not the sole criterion for enabling many of these mission critical applications.
For instance, the vast amounts of data generated by many scientific experiments needs to be analyzed, interpreted,
documented in order to create and extract knowledge bases that can be used for meaningful comparisons
of results. Further, these knowledge bases may need to be archived, annotated and made accessible
to the scientific community.Consequently, new mechanisms, interfaces are needed in the parallel
file-system to manage this data throughout its life-cycle. 

My research work has involved a good amount of survey and 
analysis of existing solutions for problems, improvements and/or a better 
solution, a detailed understanding of pros and cons of the improvements, and 
technology transfer of the results into usable products. Towards this goal, 
my work uses several implementation techniques to improve the 
performance of today's large-scale scientific applications by improving cache 
hit rates or reducing page-faults/cache misses, and leveraging weaker
consistency policies. This in turn has led to a good overall understanding
of the various components constituting the high-performance I/O software
stack. My collaboration with researchers at Argonne 
National Laboratory have also helped me get a hands-on perspective of various problems
and solutions involved in these techniques. It has helped pave a thorough 
groundwork for my future research and I expect to make the most out of it. 
\end{document}
